\documentclass[journal]{IEEEtran}


\begin{document}

\title{Reasoning in Artificial Neural Networks}

\author{%
  \IEEEauthorblockN{Vasin Srisupavanich}\\
  \IEEEauthorblockA{vs2n19@soton.ac.uk}
}

\maketitle


\begin{abstract}
The abstract goes here.
\end{abstract}


\begin{IEEEkeywords}
Deep learning, neural networks, reasoning, graph neural networks, neural-symbolic, neural turing machine
\end{IEEEkeywords}



\section{Introduction}

\IEEEPARstart{T}{he} ability to reason is one of the most important features of human intelligence. 
It allows us to understand abstract concepts and make complex decisions.
Human excels at tasks that require high level understanding, such as planning and symbol manipulation, 
while the current state of machines are limited to simpler pattern matching problems.
Incorporating reasoning ability to machines has been a long standing goal, but a very difficult challenge in the field of Artificial Intelligence. 
Solving this problem would mean a significant step toward artificial general intelligence, which will ultimately benefits humankind greatly. 
This paper reviews recent approaches in building artificial neural networks that can learn to reason, 
and overview the current state of the art results and applications from deep learning systems with reasoning capability.

\section{Background}
Over the past decade, deep learning systems have enjoyed tremendous success particularly in the area of computer vision and natural language processing.
Machines can now learn to recognize an object in an image with higher accuracy than human, synthesize a novel, and beat a world champion Go player.
However, they still struggle in tasks that involve reasoning operations. For example, deep learning system today have difficulty in
identify cause and effect (causal reasoning), and understanding what is to the left of an object (spatial-temporal reasoning).
Historically, the approach to create an AI system capable of reasoning has been from a symbolic point of view.
In a symbolic AI, knowledge is represented as symbols, rules are handcrafted by human, and reasoning is the process of inference.
However, these systems do not scale to real-world applications, as fixed symbols and rules cannot represents enough information. 
This has paved the way to the current trend of a sub-symbolic approach, which utilise artificial neural networks. 
With the goal to improve deep learning system beyond pattern matching, AI researchers have tried to combine symbolic AI with neural networks,
which has resulted in a subfield called Neural-symbolic. Apart from that, researchers also take inspiration from neuroscience. 
As human reasoning involves extracting knowledge from memory and paying attention to specific part of information, 
this has resulted in an extension of neural network in the form of memory and attention mechanism. 

\subsection{Artificial Neural Network}

\section{Main Approaches}
\subsection{Attention and Memory}

\subsection{Graph Neural Network}

\subsection{Neural-Symbolic}

\section{Discussion}

\section{Conclusion}

% \bibliographystyle{IEEEtran}
% \bibliography{references}


\end{document}


